%%%%%%%%%%%%%%%%%%%%%%%%%%%%%%%%%%%%%%%%%
% KOMA-Script Presentation
% LaTeX Template
% Version 1.1 (18/10/15)
%
% This template has been downloaded from:
% http://www.LaTeXTemplates.com
%
% Original Authors:
% Marius Hofert (marius.hofert@math.ethz.ch)
% Markus Kohm (komascript@gmx.info)
% Described in the PracTeX Journal, 2010, No. 2
%
% License:
% CC BY-NC-SA 3.0 (http://creativecommons.org/licenses/by-nc-sa/3.0/)
%
%%%%%%%%%%%%%%%%%%%%%%%%%%%%%%%%%%%%%%%%%

%----------------------------------------------------------------------------------------
%	PACKAGES AND OTHER DOCUMENT CONFIGURATIONS
%----------------------------------------------------------------------------------------

\documentclass[
paper=128mm:96mm, % The same paper size as used in the beamer class
fontsize=11pt, % Font size
pagesize, % Write page size to dvi or pdf
parskip=half-, % Paragraphs separated by half a line
]{scrartcl} % KOMA script (article)

\linespread{1.12} % Increase line spacing for readability

%------------------------------------------------
% Colors
\usepackage[dvipsnames]{xcolor}	 % Required for custom colors
% Define a few colors for making text stand out within the presentation
\definecolor{mygreen}{RGB}{44,85,17}
\definecolor{myblue}{RGB}{34,31,217}
\definecolor{mybrown}{RGB}{194,164,113}
\definecolor{myred}{RGB}{255,66,56}
% Use these colors within the presentation by enclosing text in the commands below
\newcommand*{\mygreen}[1]{\textcolor{mygreen}{#1}}
\newcommand*{\myblue}[1]{\textcolor{myblue}{#1}}
\newcommand*{\mybrown}[1]{\textcolor{mybrown}{#1}}
\newcommand*{\myred}[1]{\textcolor{myred}{#1}}
%------------------------------------------------

%------------------------------------------------
% Margins
\usepackage[ % Page margins settings
includeheadfoot,
top=3.5mm,
bottom=3.5mm,
left=5.5mm,
right=5.5mm,
headsep=6.5mm,
footskip=8.5mm
]{geometry}
%------------------------------------------------

%------------------------------------------------
% Fonts
\usepackage[T1]{fontenc}	 % For correct hyphenation and T1 encoding
\usepackage{lmodern} % Default font: latin modern font
%\usepackage{fourier} % Alternative font: utopia
%\usepackage{charter} % Alternative font: low-resolution roman font
\renewcommand{\familydefault}{\sfdefault} % Sans serif - this may need to be commented to see the alternative fonts
%------------------------------------------------

%------------------------------------------------
% Various required packages
\usepackage{amsthm} % Required for theorem environments
\usepackage{bm} % Required for bold math symbols (used in the footer of the slides)
\usepackage{graphicx} % Required for including images in figures
\usepackage{tikz} % Required for colored boxes
\usepackage{booktabs} % Required for horizontal rules in tables
\usepackage{multicol} % Required for creating multiple columns in slides
\usepackage{lastpage} % For printing the total number of pages at the bottom of each slide
\usepackage[english]{babel} % Document language - required for customizing section titles
\usepackage{microtype} % Better typography
\usepackage{tocstyle} % Required for customizing the table of contents
%------------------------------------------------

%------------------------------------------------
% Slide layout configuration
\usepackage{scrpage2} % Required for customization of the header and footer
\pagestyle{scrheadings} % Activates the pagestyle from scrpage2 for custom headers and footers
\clearscrheadfoot % Remove the default header and footer
\setkomafont{pageheadfoot}{\normalfont\color{black}\sffamily} % Font settings for the header and footer

% Sets vertical centering of slide contents with increased space between paragraphs/lists
\makeatletter
\renewcommand*{\@textbottom}{\vskip \z@ \@plus 1fil}
\newcommand*{\@texttop}{\vskip \z@ \@plus .5fil}
\addtolength{\parskip}{\z@\@plus .25fil}
\makeatother

% Remove page numbers and the dots leading to them from the outline slide
\makeatletter
\newtocstyle[noonewithdot]{nodotnopagenumber}{\settocfeature{pagenumberbox}{\@gobble}}
\makeatother
\usetocstyle{nodotnopagenumber}

\AtBeginDocument{\renewcaptionname{english}{\contentsname}{\Large Math Time}} % Change the name of the table of contents
%------------------------------------------------

%------------------------------------------------
% Header configuration - if you don't want a header remove this block
\ihead{
\hspace{-2mm}
\begin{tikzpicture}[remember picture,overlay]
\node [xshift=\paperwidth/2,yshift=-\headheight] (mybar) at (current page.north west)[rectangle,fill,inner sep=0pt,minimum width=\paperwidth,minimum height=2\headheight,top color=mygreen!64,bottom color=mygreen]{}; % Colored bar
\node[below of=mybar,yshift=3.3mm,rectangle,shade,inner sep=0pt,minimum width=128mm,minimum height =1.5mm,top color=black!50,bottom color=white]{}; % Shadow under the colored bar
shadow
\end{tikzpicture}
\color{white}\runninghead} % Header text defined by the \runninghead command below and colored white for contrast
%------------------------------------------------

%------------------------------------------------
% Footer configuration
\setlength{\footheight}{8mm} % Height of the footer
\addtokomafont{pagefoot}{\footnotesize} % Small font size for the footnote

\ifoot{% Left side
\hspace{-2mm}
\begin{tikzpicture}[remember picture,overlay]
\node [xshift=\paperwidth/2,yshift=\footheight] at (current page.south west)[rectangle,fill,inner sep=0pt,minimum width=\paperwidth,minimum height=3pt,top color=mygreen,bottom color=mygreen]{}; % Green bar
\end{tikzpicture}
\myauthor\ \raisebox{0.2mm}{$\bm{\vert}$}\ \myuni % Left side text
}

\ofoot[\pagemark/\pageref{LastPage}\hspace{-2mm}]{\pagemark/\pageref{LastPage}\hspace{-2mm}} % Right side
%------------------------------------------------

%------------------------------------------------
% Section spacing - deeper section titles are given less space due to lesser importance
%\usepackage{titlesec} % Required for customizing section spacing
%\titlespacing{\section}{1mm}{1mm}{1mm} % Lengths are: left, before, after
%\titlespacing{\subsection}{0mm}{0mm}{-1mm} % Lengths are: left, before, after
%\titlespacing{\subsubsection}{0mm}{0mm}{-2mm} % Lengths are: left, before, after
%\setcounter{secnumdepth}{0} % How deep sections are numbered, set to no numbering by default - change to 1 for numbering sections, 2 for numbering sections and subsections, etc
%------------------------------------------------

%------------------------------------------------
% Theorem style
\newtheoremstyle{mythmstyle} % Defines a new theorem style used in this template
{0.5em} % Space above
{0.5em} % Space below
{} % Body font
{} % Indent amount
{\sffamily\bfseries} % Head font
{} % Punctuation after head
{\newline} % Space after head
{\thmname{#1}\ \thmnote{(#3)}} % Head spec
	
\theoremstyle{mythmstyle} % Change the default style of the theorem to the one defined above
\newtheorem{theorem}{Theorem}[section] % Label for theorems
\newtheorem{prob}{Problem}[section]  %Label for problems 
\newtheorem{lemma}{Lemma}[section]  %Label for lemmas 
\newtheorem{defi}{Definition}[section]
\newtheorem{exmp}{Exercise}[section] %Label for examples
\newtheorem{remark}[theorem]{Remark} % Label for remarks
\newtheorem{algorithm}[theorem]{Algorithm} % Label for algorithms
\makeatletter % Correct qed adjustment
%------------------------------------------------

%------------------------------------------------
% The code for the box which can be used to highlight an element of a slide (such as a theorem)
\newcommand*{\mybox}[2]{ % The box takes two arguments: width and content
\par\noindent
\begin{tikzpicture}[mynodestyle/.style={rectangle,draw=mygreen,thick,inner sep=2mm,text justified,top color=white,bottom color=white,above}]\node[mynodestyle,at={(0.5*#1+2mm+0.4pt,0)}]{ % Box formatting
\begin{minipage}[t]{#1}
#2
\end{minipage}
};
\end{tikzpicture}
\par\vspace{-1.3em}}
%------------------------------------------------

%------------------------------------------------
% MODIFICATIONS BY JUSTIN STEVENS
%------------------------------------------------

\usepackage[nodayofweek,level]{datetime}
\usepackage{caption}
\usepackage{subcaption}
\usepackage{hyperref}
\newcommand{\pmid}{\mid\!\mid}
\usepackage{seqsplit}
\usepackage{amsfonts}
\usepackage{float} %use H to force it in place
\usepackage{amssymb} %for nmid
\usepackage{enumitem} %for itemized lists with stars
\usepackage{amsmath}
\DeclareMathOperator{\lcm}{lcm}
%\usepackage{epigraph}
\usepackage{csquotes}
\usepackage{relsize}
\newcommand{\x}{\color{red}X\color{black}}




%----------------------------------------------------------------------------------------
%	PRESENTATION INFORMATION
%----------------------------------------------------------------------------------------

\newcommand*{\mytitle}{A-Star 2016 Winter Math Camp } % Title
\newcommand*{\runninghead}{AMC Number Theory} % Running head displayed on almost all slides
\newcommand*{\myauthor}{Justin Stevens} % Presenters name(s)
\newcommand*{\mydate}{\formatdate{26}{12}{2016}} % Presentation date
\newcommand*{\myuni}{A-Star 2016 Winter Math Camp} % University or department

%----------------------------------------------------------------------------------------

\begin{document}
	

%----------------------------------------------------------------------------------------
%	TITLE SLIDE
%----------------------------------------------------------------------------------------

% Title slide - you may have to tweak a few of the numbers if you wish to make changes to the layout
\thispagestyle{empty} % No slide header and footer
\begin{tikzpicture}[remember picture,overlay] % Background box
\node [xshift=\paperwidth/2,yshift=\paperheight/2] at (current page.south west)[rectangle,fill,inner sep=0pt,minimum width=\paperwidth,minimum height=\paperheight/3,top color=mygreen,bottom color=mygreen]{}; % Change the height of the box, its colors and position on the page here
\end{tikzpicture}
% Text within the box
\begin{flushright}
\vspace{0.6cm}
\color{white}\sffamily
{\bfseries\Large\mytitle\par} % Title
\vspace{0.5cm}
\normalsize
\myauthor\par % Author name
\mydate\par % Date
\vfill
\end{flushright}



\clearpage


%----------------------------------------------------------------------------------------
%	TABLE OF CONTENTS
%----------------------------------------------------------------------------------------

\thispagestyle{empty} % No slide header and footer

\small\tableofcontents % Change the font size and print the table of contents - it may be useful to shrink the font size further if the presentation is full of sections
% To exclude sections/subsections from the table of contents, put an asterisk after \(sub)section like so: \section*{Section Name}

\clearpage

\section{Algebraic Manipulation}

In this section, we will explore several of my favourite problems involving algebraic manipulations.

\mybox{0.8\textwidth}{\begin{prob}[2000 AMC 12]  If $x,y,$ and $z$ are positive numbers satisfying $$x+\frac{1}{y}=4\:, y+\frac{1}{z}=1\:, \text{and}\: z+\frac{1}{x}=\frac73,$$ find the value of $xyz$. \end{prob} \begin{prob}[AoPS Introduction to Algebra] Let $A=x+\frac{1}{x}$ and $B=x^2+\frac{1}{x^2}$. Note that $(x+\frac{1}{x})^2=x^2+2+\frac{1}{x^2}$, therefore, $B=A^2-2$. Find formulas for $$C=x^3+\frac{1}{x^3}\:, D=x^4+\frac{1}{x^4}, E=x^5+\frac{1}{x^5}$$ in terms of $A$.  \end{prob}}

\subsection{2000 AMC 12}
\begin{proof}[Solution]
In order to get the $xyz$ term, we are motivated to multiply the $3$ equations together: \begin{eqnarray*} \left(x+\frac{1}{y}\right)\left(y+\frac{1}{z}\right)\left(z+\frac{1}{x}\right)&=&xyz+\frac{1}{xyz}+\left(x+y+z\right)+\left(\frac{1}{x}+\frac{1}{y}+\frac{1}{z}\right) \\ &=& \left(4\right)\left(1\right)\left(\frac73\right)=\frac{28}{3}. \end{eqnarray*}  

What can we do now to simplify this further?
\clearpage
We also add all $3$ of the equations: \begin{eqnarray*} \left(x+\frac{1}{y}\right)+\left(y+\frac{1}{z}\right)+\left(z+\frac{1}{x}\right)&=&4+1+\frac{7}{3}=\frac{22}{3} \\ &=& \left(x+y+z\right)+\left(\frac{1}{x}+\frac{1}{y}+\frac{1}{z}\right). \end{eqnarray*}  
Therefore, plugging this in to the first equation gives \begin{eqnarray*} xyz+\frac{1}{xyz}+\left(x+y+z\right)+\left(\frac{1}{x}+\frac{1}{y}+\frac{1}{z}\right) &=& xyz+\frac{1}{xyz}+\frac{22}{3} \\ &=&\frac{28}{3} \\ \implies \color{blue} xyz+\frac{1}{xyz} &=& \color{blue} 2. \end{eqnarray*}  

What's $xyz$ equal to then?

\clearpage

Multiply the equation through by $xyz$ and simplify: $$(xyz)^2+1=2xyz\implies (xyz)^2-2\cdot xyz+1=(xyz-1)^2=0.$$  
Therefore, $xyz=\boxed{1}$.  \end{proof}

\subsection{Exponent Mayhem} 

\begin{proof}[Solution]
	
We begin with $C$. Inspired by our method for computing $B$, we attempt to cube $x+\frac{1}{x}$: \begin{eqnarray*} (x+\frac{1}{x})^3&=&x^3+3\cdot x^2\cdot \frac{1}{x}+3\cdot x\cdot \frac{1}{x^2}+\frac{1}{x^3} \\ &=& x^3+3x+\frac{3}{x}+\frac{1}{x^3}. \end{eqnarray*} 
\clearpage

Therefore, \begin{eqnarray*} A^3&=&x^3+3\left(x+\frac{1}{x}\right)+\frac{1}{x^3} \:=\:C+3A \\ \implies \color{blue} C &=& \color{blue} A^3-3A. \end{eqnarray*}

There are two methods for finding $D$. One of them involves taking $x+\frac{1}{x}$ to the fourth power. In order to continue with this method, however, I must introduce the binomial theorem and Pascal's triangle.  

\end{proof}


\end{document}  