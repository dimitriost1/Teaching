%%%%%%%%%%%%%%%%%%%%%%%%%%%%%%%%%%%%%%%%%
% KOMA-Script Presentation
% LaTeX Template
% Version 1.1 (18/10/15)
%
% This template has been downloaded from:
% http://www.LaTeXTemplates.com
%
% Original Authors:
% Marius Hofert (marius.hofert@math.ethz.ch)
% Markus Kohm (komascript@gmx.info)
% Described in the PracTeX Journal, 2010, No. 2
%
% License:
% CC BY-NC-SA 3.0 (http://creativecommons.org/licenses/by-nc-sa/3.0/)
%
%%%%%%%%%%%%%%%%%%%%%%%%%%%%%%%%%%%%%%%%%

%----------------------------------------------------------------------------------------
%	PACKAGES AND OTHER DOCUMENT CONFIGURATIONS
%----------------------------------------------------------------------------------------

\documentclass[
paper=128mm:96mm, % The same paper size as used in the beamer class
fontsize=11pt, % Font size
pagesize, % Write page size to dvi or pdf
parskip=half-, % Paragraphs separated by half a line
]{scrartcl} % KOMA script (article)

\linespread{1.12} % Increase line spacing for readability

%------------------------------------------------
% Colors
\usepackage[dvipsnames]{xcolor}	 % Required for custom colors
% Define a few colors for making text stand out within the presentation
\definecolor{mygreen}{RGB}{44,85,17}
\definecolor{myblue}{RGB}{34,31,217}
\definecolor{mybrown}{RGB}{194,164,113}
\definecolor{myred}{RGB}{255,66,56}
% Use these colors within the presentation by enclosing text in the commands below
\newcommand*{\mygreen}[1]{\textcolor{mygreen}{#1}}
\newcommand*{\myblue}[1]{\textcolor{myblue}{#1}}
\newcommand*{\mybrown}[1]{\textcolor{mybrown}{#1}}
\newcommand*{\myred}[1]{\textcolor{myred}{#1}}
%------------------------------------------------

%------------------------------------------------
% Margins
\usepackage[ % Page margins settings
includeheadfoot,
top=3.5mm,
bottom=3.5mm,
left=5.5mm,
right=5.5mm,
headsep=6.5mm,
footskip=8.5mm
]{geometry}
%------------------------------------------------

%------------------------------------------------
% Fonts
\usepackage[T1]{fontenc}	 % For correct hyphenation and T1 encoding
\usepackage{lmodern} % Default font: latin modern font
%\usepackage{fourier} % Alternative font: utopia
%\usepackage{charter} % Alternative font: low-resolution roman font
\renewcommand{\familydefault}{\sfdefault} % Sans serif - this may need to be commented to see the alternative fonts
%------------------------------------------------

%------------------------------------------------
% Various required packages
\usepackage{amsthm} % Required for theorem environments
\usepackage{bm} % Required for bold math symbols (used in the footer of the slides)
\usepackage{graphicx} % Required for including images in figures
\usepackage{tikz} % Required for colored boxes
\usepackage{booktabs} % Required for horizontal rules in tables
\usepackage{multicol} % Required for creating multiple columns in slides
\usepackage{lastpage} % For printing the total number of pages at the bottom of each slide
\usepackage[english]{babel} % Document language - required for customizing section titles
\usepackage{microtype} % Better typography
\usepackage{tocstyle} % Required for customizing the table of contents
%------------------------------------------------

%------------------------------------------------
% Slide layout configuration
\usepackage{scrpage2} % Required for customization of the header and footer
\pagestyle{scrheadings} % Activates the pagestyle from scrpage2 for custom headers and footers
\clearscrheadfoot % Remove the default header and footer
\setkomafont{pageheadfoot}{\normalfont\color{black}\sffamily} % Font settings for the header and footer

% Sets vertical centering of slide contents with increased space between paragraphs/lists
\makeatletter
\renewcommand*{\@textbottom}{\vskip \z@ \@plus 1fil}
\newcommand*{\@texttop}{\vskip \z@ \@plus .5fil}
\addtolength{\parskip}{\z@\@plus .25fil}
\makeatother

% Remove page numbers and the dots leading to them from the outline slide
\makeatletter
\newtocstyle[noonewithdot]{nodotnopagenumber}{\settocfeature{pagenumberbox}{\@gobble}}
\makeatother
\usetocstyle{nodotnopagenumber}

\AtBeginDocument{\renewcaptionname{english}{\contentsname}{\Large Math Time}} % Change the name of the table of contents
%------------------------------------------------

%------------------------------------------------
% Header configuration - if you don't want a header remove this block
\ihead{
\hspace{-2mm}
\begin{tikzpicture}[remember picture,overlay]
\node [xshift=\paperwidth/2,yshift=-\headheight] (mybar) at (current page.north west)[rectangle,fill,inner sep=0pt,minimum width=\paperwidth,minimum height=2\headheight,top color=mygreen!64,bottom color=mygreen]{}; % Colored bar
\node[below of=mybar,yshift=3.3mm,rectangle,shade,inner sep=0pt,minimum width=128mm,minimum height =1.5mm,top color=black!50,bottom color=white]{}; % Shadow under the colored bar
shadow
\end{tikzpicture}
\color{white}\runninghead} % Header text defined by the \runninghead command below and colored white for contrast
%------------------------------------------------

%------------------------------------------------
% Footer configuration
\setlength{\footheight}{8mm} % Height of the footer
\addtokomafont{pagefoot}{\footnotesize} % Small font size for the footnote

\ifoot{% Left side
\hspace{-2mm}
\begin{tikzpicture}[remember picture,overlay]
\node [xshift=\paperwidth/2,yshift=\footheight] at (current page.south west)[rectangle,fill,inner sep=0pt,minimum width=\paperwidth,minimum height=3pt,top color=mygreen,bottom color=mygreen]{}; % Green bar
\end{tikzpicture}
\myauthor\ \raisebox{0.2mm}{$\bm{\vert}$}\ \myuni % Left side text
}

\ofoot[\pagemark/\pageref{LastPage}\hspace{-2mm}]{\pagemark/\pageref{LastPage}\hspace{-2mm}} % Right side
%------------------------------------------------

%------------------------------------------------
% Section spacing - deeper section titles are given less space due to lesser importance
%\usepackage{titlesec} % Required for customizing section spacing
%\titlespacing{\section}{1mm}{1mm}{1mm} % Lengths are: left, before, after
%\titlespacing{\subsection}{0mm}{0mm}{-1mm} % Lengths are: left, before, after
%\titlespacing{\subsubsection}{0mm}{0mm}{-2mm} % Lengths are: left, before, after
%\setcounter{secnumdepth}{0} % How deep sections are numbered, set to no numbering by default - change to 1 for numbering sections, 2 for numbering sections and subsections, etc
%------------------------------------------------

%------------------------------------------------
% Theorem style
\newtheoremstyle{mythmstyle} % Defines a new theorem style used in this template
{0.5em} % Space above
{0.5em} % Space below
{} % Body font
{} % Indent amount
{\sffamily\bfseries} % Head font
{} % Punctuation after head
{\newline} % Space after head
{\thmname{#1}\ \thmnote{(#3)}} % Head spec
	
\theoremstyle{mythmstyle} % Change the default style of the theorem to the one defined above
\newtheorem{theorem}{Theorem}[section] % Label for theorems
\newtheorem{prob}{Problem}[section]  %Label for problems 
\newtheorem{lemma}{Lemma}[section]  %Label for lemmas 
\newtheorem{defi}{Definition}[section]
\newtheorem{exmp}{Exercise}[section] %Label for examples
\newtheorem{remark}[theorem]{Remark} % Label for remarks
\newtheorem{algorithm}[theorem]{Algorithm} % Label for algorithms
\makeatletter % Correct qed adjustment
%------------------------------------------------

%------------------------------------------------
% The code for the box which can be used to highlight an element of a slide (such as a theorem)
\newcommand*{\mybox}[2]{ % The box takes two arguments: width and content
\par\noindent
\begin{tikzpicture}[mynodestyle/.style={rectangle,draw=mygreen,thick,inner sep=2mm,text justified,top color=white,bottom color=white,above}]\node[mynodestyle,at={(0.5*#1+2mm+0.4pt,0)}]{ % Box formatting
\begin{minipage}[t]{#1}
#2
\end{minipage}
};
\end{tikzpicture}
\par\vspace{-1.3em}}
%------------------------------------------------

%------------------------------------------------
% MODIFICATIONS BY JUSTIN STEVENS
%------------------------------------------------

\usepackage[nodayofweek,level]{datetime}
\usepackage{caption}
\usepackage{subcaption}
\usepackage{hyperref}
\newcommand{\pmid}{\mid\!\mid}
\usepackage{seqsplit}
\usepackage{amsfonts}
\usepackage{float} %use H to force it in place
\usepackage{amssymb} %for nmid
\usepackage{enumitem} %for itemized lists with stars
\usepackage{amsmath}
\DeclareMathOperator{\lcm}{lcm}
%\usepackage{epigraph}
\usepackage{csquotes}
\usepackage{relsize}
\newcommand{\x}{\color{red}X\color{black}}




\lhead{Telescoping Series}
\title{ARML Lecture:  Telescoping Series}
\begin{document}
\begin{center}
\HRule \\[0.4cm]
{ \huge \bfseries ARML: Telescoping Series}\\[0.4cm] % Title of your document
\HRule \\[1.5cm]
\begin{minipage}{0.4\textwidth}
\begin{flushleft} \large
\emph{Authors}\\
Justin \textsc{Stevens} \newline
\end{flushleft}
\end{minipage}
~
\begin{minipage}{0.4\textwidth}
\begin{flushright} \large
Winter 2015
\end{flushright}
\end{minipage}\\[0.5cm]
\end{center}

\section{Lecture}

With certain sums/products, the majority of the terms will cancel which helps to simplify calculations.  Notation used throughout the document:

$$\prod_{i=1}^{n}a_i=a_1\times a_2\times a_3\times \cdots \times a_n$$
$$\sum_{i=1}^{n}a_i=a_1+a_2+a_3+\cdots+a_n$$

\begin{exmp}[Mathcounts]  Evaluate the product $$\left(1+\frac12\right)\left(1+\frac13\right)\left(1+\frac14\right)\left(1+\frac15\right)\left(1+\frac16\right)\left(1+\frac17\right)$$ \end{exmp}
\begin{soln}  
The product is equivalent to $$\left(\frac32\right)\left(\frac43\right)\left(\frac54\right)\left(\frac65\right)\left(\frac76\right)\left(\frac87\right)=\frac{8}{2}=4$$ after cancellation   \end{soln}

\begin{exmp}  Simplify the product $$\left(1-\frac13\right)\left(1-\frac14\right)\left(1-\frac15\right)\cdots\left(1-\frac1n\right)$$ \end{exmp}
\begin{soln}
\begin{eqnarray*}  \left(1-\frac13\right)\left(1-\frac14\right)\left(1-\frac15\right)\cdots \left(1-\frac1n\right) &=& \left(\frac23\right)\left(\frac34\right)\left(\frac45\right)\cdots\left(\frac{n-1}{n}\right) \\ 
&=& \left(\frac{2}{\cancel{3}}\right)\left(\frac{\cancel{3}}{\cancel{4}}\right)\left(\frac{\cancel{4}}{\cancel{5}}\right)\cdots \left(\frac{\cancel{n-1}}n \right) \\ &=& \frac{2}{n} \end{eqnarray*}
\end{soln}

\begin{exmp}   Evaluate $\displaystyle \prod_{k=2}^{n}\left(1-\frac{1}{k^2}\right)$  \end{exmp}
\begin{soln}  
\begin{eqnarray*} \prod_{k=2}^{n}\left(1-\frac{1}{k^2}\right) &=& \prod_{k=2}^{n}\frac{(k+1)(k-1)}{k^2} \\ &=& \left(\prod_{k=2}^{n}\frac{k+1}{k}\right)\left(\prod_{k=2}^{n}\frac{k-1}{k}\right) \\ &=& \left(\frac{n+1}{2}\right)\left(\frac{1}{n}\right)=\frac{n+1}{2n} \end{eqnarray*}  \end{soln}

\begin{exmp}[AMC 12] Let $T_n=1+2+3+\cdots+n$ and $$P_n=\frac{T_2}{T_2-1}\cdot \frac{T_3}{T_3-1}\cdot \frac{T_4}{T_4-1}\cdots \frac{T_n}{T_n-1}$$ for $n=2,3,4,\cdots$.  Find $P_{1991}$. \end{exmp}  
\begin{soln} 
Notice that $T_n=\frac{n(n+1)}{2}$ and $$T_n-1=\frac{n(n+1)}{2}-1=\frac{n^2+n-2}{2}=\frac{(n+2)(n-1)}{2}$$.  Therefore the product which we want to evaluate is equivalent to
\begin{eqnarray*} P_n&=&\left(\prod_{i=2}^{1991}\frac{i}{i+2}\right)\left(\prod_{i=2}^{1991}\frac{i+1}{i-1}\right) \\ &=& \frac{2\times 3}{1992\times 1993}\times \left(\frac{1991\times 1992}{1\times 2}\right) \\ &=& \frac{3\times 1991}{1993}=\frac{5973}{1993} \end{eqnarray*} \end{soln}

\begin{exmp}  Evaluate the sum $$\frac{1}{1\times 2}+\frac{1}{2\times 3}+\frac{1}{3\times 4}+\frac{1}{4\times 5}+\cdots +\frac{1}{99\times 100}$$ \end{exmp}
\begin{soln}  
Notice that $\displaystyle \frac{1}{n(n+1)}=\frac{1}{n}-\frac{1}{n+1}$, therefore the sum is equivalent to 
$$\left(\frac11-\frac12\right)+\left(\frac12-\frac13\right)+\left(\frac13-\frac14\right)+\cdots+\left(\frac{1}{99}-\frac{1}{100}\right)= \frac11-\frac{1}{100}=\frac{99}{100}$$

\end{soln}



\section{Problem Solving}

Here is a set of problems involving telescoping series.  If you have any questions or want hints on any of these questions please feel free to ask me!  

\begin{prob}[AHSME]  Find the sum $\displaystyle \frac{1}{1\cdot 3}+\frac{1}{3\cdot 5}+\cdots+\frac{1}{(2n-1)(2n+1)}+\cdots+\frac{1}{255\times 257}$ \end{prob}
\begin{prob}  Find the product $\displaystyle  \prod_{n=1}^{20}\left(1+\frac{2n+1}{n^2}\right)$.  \end{prob}  
\begin{prob}  Consider the sequence $1, -2, 3, -4, 5, -6, \cdots$ whose $n$th term is $(-1)^{n+1}\cdot n$.  What is the average of the first $200$ terms of the sequence?  \end{prob}  
\begin{prob}[HMMT]  Evaluate $1\cdot 2-2\cdot 3+3\cdot 4-4\cdot 5+\cdots+2001\cdot 2002$.  \end{prob}
\begin{prob}[Mandelbrot]  Calculate $$\prod_{n=1}^{13}\frac{n(n+2)}{(n+4)^2}$$ \end{prob}
\begin{prob}[AHSME]  Calculate $$ T=\frac{1}{3-\sqrt{8}}-\frac{1}{\sqrt{8}-\sqrt{7}}+\frac{1}{\sqrt{7}-\sqrt{6}}-\frac{1}{\sqrt{6}-\sqrt{5}}+\frac{1}{\sqrt{5}-2}. $$ \end{prob}
\begin{prob}  Find $$\frac{1}{1+\sqrt2}+\frac{1}{\sqrt2+\sqrt3}+\frac{1}{\sqrt3+\sqrt4}+\cdots+\frac{1}{\sqrt{99}+\sqrt{100}}$$ \end{prob}
\begin{prob}  Find the sum $\frac{1}{2!}+\frac{2}{3!}+\frac{3}{4!}+\cdots+\frac{n-1}{n!}$ \end{prob}  
\begin{prob}[AIME]  Let $ \displaystyle x=\frac{4}{(\sqrt{5}+1)(\sqrt[4]{5}+1)(\sqrt[8]{5}+1)(\sqrt[16]{5}+1)}. $ Find $ (x+1)^{48}. $ \end{prob}  
\begin{prob}  Find the integer closest to $\displaystyle 1000\sum_{n=3}^{1000}\frac{1}{n^2-4}$.  \end{prob}  
\begin{prob}    Evaluate $\displaystyle \sum_{k=2}^{n}k!(k^2+k+1)$  \end{prob}  
\begin{prob}[Mandelbrot]  Compute the product $$\frac{\left(1998^2-1996^2\right)\left(1998^2-1995^2\right)\cdots \left(1998^2-0^2\right)}{\left(1997^2-1996^2\right)\left(1997^2-1995^2\right)\cdots \left(1997^2-0^2\right)}$$  \end{prob}  
\begin{prob}[USAMTS]  Determine the value of $$S=\sqrt{1+\frac{1}{1^2}+\frac{1}{2^2}}+\sqrt{1+\frac{1}{2^2}+\frac{1}{3^2}}+\cdots+\sqrt{1+\frac{1}{n^2}+\frac{1}{(n+1)^2}}+\cdots+\sqrt{1+\frac{1}{1999^2}+\frac{1}{2000^2}}$$  [Hint:  This problem is very difficult.  Try expressing each of the radicals in term of $n$] \end{prob}
\begin{prob}  Evaluate the infinite product $\displaystyle \prod_{n=2}^{\infty}\left(\frac{n^3-1}{n^3+1}\right)$  [Hint:  Factor and write out the first few terms]  \end{prob}  



\end{document}