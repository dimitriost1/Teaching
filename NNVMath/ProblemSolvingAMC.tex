%%%%%%%%%%%%%%%%%%%%%%%%%%%%%%%%%%%%%%%%%
% KOMA-Script Presentation
% LaTeX Template
% Version 1.1 (18/10/15)
%
% This template has been downloaded from:
% http://www.LaTeXTemplates.com
%
% Original Authors:
% Marius Hofert (marius.hofert@math.ethz.ch)
% Markus Kohm (komascript@gmx.info)
% Described in the PracTeX Journal, 2010, No. 2
%
% License:
% CC BY-NC-SA 3.0 (http://creativecommons.org/licenses/by-nc-sa/3.0/)
%
%%%%%%%%%%%%%%%%%%%%%%%%%%%%%%%%%%%%%%%%%

%----------------------------------------------------------------------------------------
%	PACKAGES AND OTHER DOCUMENT CONFIGURATIONS
%----------------------------------------------------------------------------------------

\documentclass[
paper=128mm:96mm, % The same paper size as used in the beamer class
fontsize=11pt, % Font size
pagesize, % Write page size to dvi or pdf
parskip=half-, % Paragraphs separated by half a line
]{scrartcl} % KOMA script (article)

\linespread{1.12} % Increase line spacing for readability

%------------------------------------------------
% Colors
\usepackage[dvipsnames]{xcolor}	 % Required for custom colors
% Define a few colors for making text stand out within the presentation
\definecolor{mygreen}{RGB}{44,85,17}
\definecolor{myblue}{RGB}{34,31,217}
\definecolor{mybrown}{RGB}{194,164,113}
\definecolor{myred}{RGB}{255,66,56}
% Use these colors within the presentation by enclosing text in the commands below
\newcommand*{\mygreen}[1]{\textcolor{mygreen}{#1}}
\newcommand*{\myblue}[1]{\textcolor{myblue}{#1}}
\newcommand*{\mybrown}[1]{\textcolor{mybrown}{#1}}
\newcommand*{\myred}[1]{\textcolor{myred}{#1}}
%------------------------------------------------

%------------------------------------------------
% Margins
\usepackage[ % Page margins settings
includeheadfoot,
top=3.5mm,
bottom=3.5mm,
left=5.5mm,
right=5.5mm,
headsep=6.5mm,
footskip=8.5mm
]{geometry}
%------------------------------------------------

%------------------------------------------------
% Fonts
\usepackage[T1]{fontenc}	 % For correct hyphenation and T1 encoding
\usepackage{lmodern} % Default font: latin modern font
%\usepackage{fourier} % Alternative font: utopia
%\usepackage{charter} % Alternative font: low-resolution roman font
\renewcommand{\familydefault}{\sfdefault} % Sans serif - this may need to be commented to see the alternative fonts
%------------------------------------------------

%------------------------------------------------
% Various required packages
\usepackage{amsthm} % Required for theorem environments
\usepackage{bm} % Required for bold math symbols (used in the footer of the slides)
\usepackage{graphicx} % Required for including images in figures
\usepackage{tikz} % Required for colored boxes
\usepackage{booktabs} % Required for horizontal rules in tables
\usepackage{multicol} % Required for creating multiple columns in slides
\usepackage{lastpage} % For printing the total number of pages at the bottom of each slide
\usepackage[english]{babel} % Document language - required for customizing section titles
\usepackage{microtype} % Better typography
\usepackage{tocstyle} % Required for customizing the table of contents
%------------------------------------------------

%------------------------------------------------
% Slide layout configuration
\usepackage{scrpage2} % Required for customization of the header and footer
\pagestyle{scrheadings} % Activates the pagestyle from scrpage2 for custom headers and footers
\clearscrheadfoot % Remove the default header and footer
\setkomafont{pageheadfoot}{\normalfont\color{black}\sffamily} % Font settings for the header and footer

% Sets vertical centering of slide contents with increased space between paragraphs/lists
\makeatletter
\renewcommand*{\@textbottom}{\vskip \z@ \@plus 1fil}
\newcommand*{\@texttop}{\vskip \z@ \@plus .5fil}
\addtolength{\parskip}{\z@\@plus .25fil}
\makeatother

% Remove page numbers and the dots leading to them from the outline slide
\makeatletter
\newtocstyle[noonewithdot]{nodotnopagenumber}{\settocfeature{pagenumberbox}{\@gobble}}
\makeatother
\usetocstyle{nodotnopagenumber}

\AtBeginDocument{\renewcaptionname{english}{\contentsname}{\Large Math Time}} % Change the name of the table of contents
%------------------------------------------------

%------------------------------------------------
% Header configuration - if you don't want a header remove this block
\ihead{
\hspace{-2mm}
\begin{tikzpicture}[remember picture,overlay]
\node [xshift=\paperwidth/2,yshift=-\headheight] (mybar) at (current page.north west)[rectangle,fill,inner sep=0pt,minimum width=\paperwidth,minimum height=2\headheight,top color=mygreen!64,bottom color=mygreen]{}; % Colored bar
\node[below of=mybar,yshift=3.3mm,rectangle,shade,inner sep=0pt,minimum width=128mm,minimum height =1.5mm,top color=black!50,bottom color=white]{}; % Shadow under the colored bar
shadow
\end{tikzpicture}
\color{white}\runninghead} % Header text defined by the \runninghead command below and colored white for contrast
%------------------------------------------------

%------------------------------------------------
% Footer configuration
\setlength{\footheight}{8mm} % Height of the footer
\addtokomafont{pagefoot}{\footnotesize} % Small font size for the footnote

\ifoot{% Left side
\hspace{-2mm}
\begin{tikzpicture}[remember picture,overlay]
\node [xshift=\paperwidth/2,yshift=\footheight] at (current page.south west)[rectangle,fill,inner sep=0pt,minimum width=\paperwidth,minimum height=3pt,top color=mygreen,bottom color=mygreen]{}; % Green bar
\end{tikzpicture}
\myauthor\ \raisebox{0.2mm}{$\bm{\vert}$}\ \myuni % Left side text
}

\ofoot[\pagemark/\pageref{LastPage}\hspace{-2mm}]{\pagemark/\pageref{LastPage}\hspace{-2mm}} % Right side
%------------------------------------------------

%------------------------------------------------
% Section spacing - deeper section titles are given less space due to lesser importance
%\usepackage{titlesec} % Required for customizing section spacing
%\titlespacing{\section}{1mm}{1mm}{1mm} % Lengths are: left, before, after
%\titlespacing{\subsection}{0mm}{0mm}{-1mm} % Lengths are: left, before, after
%\titlespacing{\subsubsection}{0mm}{0mm}{-2mm} % Lengths are: left, before, after
%\setcounter{secnumdepth}{0} % How deep sections are numbered, set to no numbering by default - change to 1 for numbering sections, 2 for numbering sections and subsections, etc
%------------------------------------------------

%------------------------------------------------
% Theorem style
\newtheoremstyle{mythmstyle} % Defines a new theorem style used in this template
{0.5em} % Space above
{0.5em} % Space below
{} % Body font
{} % Indent amount
{\sffamily\bfseries} % Head font
{} % Punctuation after head
{\newline} % Space after head
{\thmname{#1}\ \thmnote{(#3)}} % Head spec
	
\theoremstyle{mythmstyle} % Change the default style of the theorem to the one defined above
\newtheorem{theorem}{Theorem}[section] % Label for theorems
\newtheorem{prob}{Problem}[section]  %Label for problems 
\newtheorem{lemma}{Lemma}[section]  %Label for lemmas 
\newtheorem{defi}{Definition}[section]
\newtheorem{exmp}{Exercise}[section] %Label for examples
\newtheorem{remark}[theorem]{Remark} % Label for remarks
\newtheorem{algorithm}[theorem]{Algorithm} % Label for algorithms
\makeatletter % Correct qed adjustment
%------------------------------------------------

%------------------------------------------------
% The code for the box which can be used to highlight an element of a slide (such as a theorem)
\newcommand*{\mybox}[2]{ % The box takes two arguments: width and content
\par\noindent
\begin{tikzpicture}[mynodestyle/.style={rectangle,draw=mygreen,thick,inner sep=2mm,text justified,top color=white,bottom color=white,above}]\node[mynodestyle,at={(0.5*#1+2mm+0.4pt,0)}]{ % Box formatting
\begin{minipage}[t]{#1}
#2
\end{minipage}
};
\end{tikzpicture}
\par\vspace{-1.3em}}
%------------------------------------------------

%------------------------------------------------
% MODIFICATIONS BY JUSTIN STEVENS
%------------------------------------------------

\usepackage[nodayofweek,level]{datetime}
\usepackage{caption}
\usepackage{subcaption}
\usepackage{hyperref}
\newcommand{\pmid}{\mid\!\mid}
\usepackage{seqsplit}
\usepackage{amsfonts}
\usepackage{float} %use H to force it in place
\usepackage{amssymb} %for nmid
\usepackage{enumitem} %for itemized lists with stars
\usepackage{amsmath}
\DeclareMathOperator{\lcm}{lcm}
%\usepackage{epigraph}
\usepackage{csquotes}
\usepackage{relsize}
\newcommand{\x}{\color{red}X\color{black}}





\title{ARML Lecture:  Problem Solving in AMC}
\begin{document}
	
\thispagestyle{empty}
	
    \begin{center}
    	\vspace*{1cm}
    	
    	\Huge
    	\textbf{Problem Solving Strategies in AMC Contests}
    	
    	\vspace{0.5cm}
    	\LARGE
    	A Collection of my Favourite Problems
    	
    	\vspace{1.5cm}
    	
    	\textbf{Justin Stevens}
    
    	
  
    	
    \end{center}

\section{Algebra}

The first few problems which we will explore in this section all invoke finding symmetry in systems of equations.  
\begin{exmp}(AMC 12)  If $x,y,$ and $z$ are positive numbers satisfying $$x+\frac{1}{y}=4, \text{ } y+\frac{1}{z}=1, \text{ and } z+\frac{1}{x}=\frac{7}{3},$$ find the value of $xyz$.  \end{exmp} 
\begin{soln}  We want to find the product $xyz$, therefore, we think to multiply the three equations, and see where that goes.  Multiplying the three equations, we find that \begin{eqnarray*}\left(x+\frac{1}{y}\right)\left(y+\frac{1}{z}\right)\left(z+\frac{1}{x}\right)&=&xyz+\frac{xy}{x}+\frac{xz}{z}+\frac{x}{zx}+\frac{yz}{y}+\frac{y}{yx}+\frac{z}{yz}+\frac{1}{xyz} \\ &=& xyz+\left(x+y+z\right)+\left(\frac{1}{x}+\frac{1}{y}+\frac{1}{z}\right)+\frac{1}{xyz} \\ &=& 4\cdot 1 \cdot \frac73=\frac{28}{3}. \end{eqnarray*}  

Our goal is to find $xyz$, however, there are a lot of terms in the above product which we don't know how to compute directly.  However, we note that each of these terms appears exactly once in the above equations, therefore, we now think to add the original three equations: $$\left(x+\frac{1}{y}\right)+\left(y+\frac{1}{z}\right)+\left(z+\frac{1}{x}\right)=4+1+\frac{7}{3}=\frac{22}{3}$$

Substituting this into the first equation gives us $$xyz+\frac{22}{3}+\frac{1}{xyz}=\frac{28}{3}\implies xyz+\frac{1}{xyz}=2.$$

Finally, multiplying by $xyz$ and re-arranging shows that the above equation is equivalent to $$(xyz-1)^2=0\implies xyz=\boxed{1}.$$

\end{soln} 

\begin{exmp}(Purple Comet) Let $a,b,$ and $c$ be non-zero real numbers such that $$\frac{ab}{a+b}=3, \text{ } \frac{bc}{b+c}=4, \text{ and } \frac{ca}{c+a}=5.$$  Find $\displaystyle \frac{abc}{ab+bc+ca}$.  \end{exmp}
\begin{soln}  From our success in the previous problem, we think that possibly multiplying the three equations may be beneficial:  $$\left(\frac{ab}{a+b}\right)\left(\frac{bc}{b+c}\right)\left(\frac{ca}{c+a}\right)=\frac{(abc)^2}{a^2b+ab^2+a^2c+c^2a+b^2c+c^2b+2abc}$$

While we could try to manipulate the denominator in some way, this doesn't look particularly promising, therefore, we begin at the drawing board.  Since we are dealing with fractions, we think that this may have something to do with reciprocals.  Note that $$\frac{1}{a}+\frac{1}{b}=\frac{a+b}{ab},$$ which is exactly the reciprocal of the first expression.  Therefore, $\frac{1}{a}+\frac{1}{b}=\frac13$.  Using the same idea for the other equation, we find that \begin{eqnarray} \label{rec:1} \frac{1}{a}+\frac{1}{b}=\frac13; \quad \frac{1}{b}+\frac{1}{c}=\frac14; \quad \frac{1}{c}+\frac{1}{a}=\frac15. \end{eqnarray}

Now, taking the reciprocal of our desired expression,  $\displaystyle \frac{ab+bc+ca}{abc}=\frac{1}{c}+\frac{1}{a}+\frac{1}{b}.$ \newline We therefore want to find the sum of the reciprocals of $a,b,c$.  In order to do this, summing up the equations from \ref{rec:1}, we see that $$\frac{2}{a}+\frac{2}{b}+\frac{2}{c}=\frac13+\frac14+\frac15=\frac{47}{60}\implies \frac{1}{a}+\frac{1}{b}+\frac{1}{c}=\frac{47}{120}.$$

Finally, $$\frac{abc}{ab+bc+ca}=\frac{1}{\frac{ab+bc+ca}{abc}}=\frac{1}{\frac1a+\frac1b+\frac1c}=\boxed{\frac{120}{47}}.$$




	
	\end{soln}

\begin{exmp} Find $\displaystyle x^5+\frac{1}{x^5}$ in terms of $\displaystyle x+\frac{1}{x}$.  \end{exmp}
\begin{soln}  There are several equally valid ways to approach this problem.  Define $\displaystyle A_n=x^n+\frac{1}{x^n}$.  In order to find $A_2$, we simply square $A_1$:  $$\left(x+\frac{1}{x}\right)^2=x^2+2\cdot x\cdot \frac{1}{x}+\frac{1}{x^2}=x^2+2+\frac{1}{x^2}\implies A_2=A_1^2-2.$$

In order to find $A_3$ and $A_4$, we will try to use the previous value in order to recursively find these.  For instance, to find $A_3$, we would simply multiply $A_2$ by $A_1$ in two different ways.     \begin{eqnarray*} \left(x^2+\frac{1}{x^2}\right)\left(x+\frac{1}{x}\right)&=&\left(A_1^2-2\right)\left(A_1\right)=A_1^3-2A_1 \\  &=& x^3+x+\frac{1}{x}+\frac{1}{x^3}=A_3+A_1   \end{eqnarray*}
Equating the final two terms of both equations, we see that $$A_3+A_1=A_1^3-2A_1\implies A_3=A_1^3-3A_1.$$

Similarly, in order to find $A_4$, we multiply $A_3$ by $A_1$ in two different ways: 
\begin{eqnarray*}  \left(x^3+\frac{1}{x^3}\right)\left(x+\frac{1}{x}\right)&=&\left(A_1^3-3A_1\right)\left(A_1\right)=A_1^4-3A_1^2 \\ &=& x^4+x^2+\frac{1}{x^2}+\frac{1}{x^4}=A_4+A_2 \end{eqnarray*}
Equating the final two terms of both equations, we see that  $$A_4=\left(A_1^4-3A_1^2\right)-A_2=\left(A_1^4-3A_1^2\right)-\left(A_1^2-2\right)=A_1^4-4A_1^2+2.$$

Once we have the value of $A_3$ and $A_4$, we simply have to multiply $A_4$ by $A_1$ in order to find $A_5$:  

\begin{eqnarray*}  
	\left(x^4+\frac{1}{x^4}\right)\left(x+\frac1x\right)&=&\left(A_1^4-4A_1^2+2\right)\left(A_1\right)=A_1^5-4A_1^3+2A_1 \\ &=& x^5+x^3+\frac{1}{x^3}+\frac{1}{x^5}=A_5+A_3 \end{eqnarray*}

Therefore, equating the final two terms, we see that $$A_5=\left(A_1^5-4A_1^3+2A_1\right)-A_3=\left(A_1^5-4A_1^3+2A_1\right)-\left(A_1^3-3A_1\right)=\boxed{A_1^5-5A_1^3+5A_1}.$$

\end{soln}

\begin{comment}  

Another way to find $A_3$ and $A_4$ are by expanding:

When we cube $A_1$, we find: $$\left(x+\frac{1}{x}\right)^3=x^3+3x^2 \frac{1}{x}+3x \frac{1}{x^2}+\frac{1}{x^3}=x^3+3x+\frac{3}{x}+\frac{1}{x^3}\implies A_3=A_1^3-3A_1.$$

When we square $A_2$, we find $$\left(x^2+\frac{1}{x^2}\right)^2=x^4+2\cdot x^2\cdot \frac{1}{x^2}+\frac{1}{x^4}=x^4+2+\frac{1}{x^4}.$$
Remembering that $A_2=A_1^2-2$ from before, we now have \begin{eqnarray*}  A_4=x^4+\frac{1}{x^4}&=&\left(x^2+\frac{1}{x^2}\right)^2-2 \\ &=&\left(A_1^2-2\right)^2-2=A_1^4-4A_1^2+2. \end{eqnarray*}  \end{comment}  

The next few problems all involve functions.  

\begin{exmp}(Mandelbrot)  Let $f(x)$ be a function with domain $\mathbb{N}$.  If $a$ and $b$ are positive integers such that when $a+b=2^n$ for positive integer $n$, then $f(a)+f(b)=n^2$.  Find the value of $f(2002)$.  \end{exmp}
\begin{soln}  The condition in the problem statement initially seems a bit weird, therefore, we try plugging in some small values of $a$ and $b$ to see if we can figure out some properties of the function.  When $a=b=1$, we have $n=1$, therefore $f(1)+f(1)=1\implies f(1)=\frac12$.  When $a=b=2$, we have $n=2$, therefore $f(2)+f(2)=4\implies f(2)=2$.  We see that we can calculate $f(2^k)$ in general using this method. However, the value which we want, $2002$, unfortunately is not a power of $2$.  We therefore have to think of another method to calculate $f(2002)$.  
	
One way to begin is by finding a power of $2$ close to $2002$.  The nearest power of $2$ is $2048=2^{11}$.  Therefore, if we set $a=2002$ and $b=46$ in the original statement, we see that $$2002+46=2^{11}\implies f(2002)+f(46)=11^2=121.$$
We have now reduced the problem down to finding the value of $f(46)$.  This approach seems promising, therefore, we try it again.  The closest power of $2$ to $46$ is $64$, therefore, when $a=46$ and $b=18$,  $$46+18=2^6\implies f(46)+f(18)=6^2=36.$$
Repeating this process a few more times gives \begin{eqnarray*}  18+14=2^5\implies f(18)+f(14)&=&5^2=25 \\ 14+2=2^4\implies f(14)+f(2)&=& 4^2=16. \end{eqnarray*}

However, we know the value of $f(2)$ from above; it's simply $2$!  Therefore, substituting this into the final equation gives $f(14)=16-2=14$.  Continuing this chain, we find $$f(18)=25-f(14)=11; \quad f(46)=36-f(18)=25; \quad f(2002)=121-f(46)=\boxed{96}.$$  \end{soln}

\begin{exmp}(AMC 12)  Let $f$ be a function for which $f(x/3) = x^2 + x + 1$. Find the sum of all values of $z$ for which $f(3z) = 7$. \end{exmp}

\begin{exmp}(HMMT)  Define $a\star b=ab+a+b$ for all integers $a$ and $b$.  Evaluate $$1\star(2\star(3\star(4\star\cdots(99\star100)\cdots))).$$ \end{exmp}
\begin{soln}  
We use Simon's Favourite Factoring Trick.  Note that $(a+1)(b+1)=ab+a+b+1$.  Therefore, we have $$a\star b=(a+1)(b+1)-1.$$

Using this property, we have $99\star 100=100\times 101-1$.  Now, looking at the next term, \begin{eqnarray*} 98\star(99\star 100)=98\star(100\times 101-1)&=&(98+1)(100\times 101-1+1)-1 \\ &=&99\times 100\times 101-1. \end{eqnarray*}  
It looks like there may be a pattern.  We investigate further, and find that \begin{eqnarray*} 97\star\left(98\star(99\star 100)\right)=97\star(99\times 100\times 101-1)&=&\left(97+1\right)\left(99\times 100\times 101-1+1\right)-1 \\ &=&98\times 99\times 100\times 101-1. \end{eqnarray*}
The pattern will continue to hold, because we see that we always add $1$ to $b$ in the product.  Therefore, $$1\star(2\star(3\star(4\star\cdots(99\star100)\cdots)))=2\times 3\times 4\times \cdots \times 100\times 101-1=\boxed{101!-1}.$$
\end{soln}
The next few problems all involve polynomials.

\begin{exmp}  Let the polynomial $p(x)=x^3-3x^2-5x+2$ have roots $r,s,t$.  Find the polynomial with roots 
	\begin{itemize}
		\item $r+1$, $s+1$, and $t+1$.
		\item $3r$, $3s$, and $3t$.
		\item $\frac{1}{r}$, $\frac{1}{s}$, and $\frac{1}{t}$.
	\end{itemize}
\end{exmp}

The next few problems all involve infinite sums.

\begin{exmp}(Vishal Arul)  Define $A(n)=1+2+3+\cdots+n$.  Compute the value of the infinite sum $$\sum_{i=1}^{\infty}\left(\frac{A(n)}{3^n}\right).$$ \end{exmp}




\end{document}