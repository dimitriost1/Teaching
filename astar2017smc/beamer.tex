
%
% Choose how your presentation looks.
%
% For more themes, color themes and font themes, see:
% http://deic.uab.es/~iblanes/beamer_gallery/index_by_theme.html
%
\mode<presentation>
  \usetheme{Madrid}      % or try Darmstadt, Madrid, Warsaw, ...
  \usecolortheme{beaver} % or try albatross, beaver, crane, ...
  %\usefonttheme{serif}  % or try serif, structurebold, ...
  \setbeamertemplate{navigation symbols}{}
  \setbeamertemplate{caption}[numbered]
  
\newcounter{example}
\newenvironment<>{exmp}[1][]{%
    \refstepcounter{example}\par\medskip
  \alert{\textbf{Example~\theexample.} }}{}
  
\newenvironment<>{thm}[1][]{%
 \par\medskip
\textbf{\textcolor{MidnightBlue}{\sffamily Theorem.  }}}{}
\newenvironment<>{exmp*}[1][]{%
\par\medskip
  \alert{\textbf{Example.} }}{}
  
\makeatletter
\newenvironment<>{myproof}[1][\proofname]{%
  \par
  \def\insertproofname{#1.}%
  \pushQED{\qed}
  \textit{\insertproofname}  }
{}
\makeatother

\setbeamertemplate{headline}{}
\setbeamercovered{transparent}

\usepackage{etoolbox}
\makeatletter
\patchcmd{\beamer@continueautobreak}{\frametitle}{\beamer@gobbleoptional}{}{\errmessage{failed to patch}}
\patchcmd{\beamer@continueautobreak}{\framesubtitle}{\beamer@gobbleoptional}{}{\errmessage{failed to patch}}
\makeatother

\makeatother
\setbeamertemplate{footline}
{
  \leavevmode%
  \hbox{%
  \begin{beamercolorbox}[wd=.4\paperwidth,ht=2.25ex,dp=1ex,center]{author in head/foot}%
    \usebeamerfont{author in head/foot}\insertshortauthor
  \end{beamercolorbox}%
  \begin{beamercolorbox}[wd=.6\paperwidth,ht=2.25ex,dp=1ex,center]{title in head/foot}%
    \usebeamerfont{title in head/foot}\insertshorttitle\hspace*{3em}
    \insertframenumber{} / \inserttotalframenumber\hspace*{1ex}
  \end{beamercolorbox}}%
  \vskip0pt%
}
\makeatletter
\setbeamertemplate{navigation symbols}{}

\colorlet{LightSpringGreen}{White!70!SpringGreen}
 \usepackage{transparent}
 \newcommand{\semitransp}[2][35]{\color{fg!#1}#2}
\usepackage[T1]{fontenc}	 % For correct hyphenation and T1 encoding
\usepackage{lmodern} % Default font: latin modern font
%\usepackage{fourier} % Alternative font: utopia
%\usepackage{charter} % Alternative font: low-resolution roman font
\renewcommand{\familydefault}{\sfdefault} % Sans serif - this may need to be commented to see the alternative fonts

\usepackage[english]{babel}
\usepackage[utf8x]{inputenc}
\usepackage{xcolor}
\usepackage{listings}

\lstset
{
    language=[LaTeX]TeX,
    breaklines=true,
    basicstyle=\tt\scriptsize,
    %commentstyle=\color{green}
    keywordstyle=\color{blue},
    %stringstyle=\color{black}
    identifierstyle=\color{magenta},
}


\AtBeginSection[]
{
  \begin{frame}<beamer>
    \frametitle{Outline}
    \tableofcontents[currentsection, hideothersubsections]
  \end{frame}
}





\usepackage{listings}
\usepackage{setspace}
\definecolor{Code}{rgb}{0,0,0}
\definecolor{Decorators}{rgb}{0.5,0.5,0.5}
\definecolor{Numbers}{rgb}{0.5,0,0}
\definecolor{MatchingBrackets}{rgb}{0.25,0.5,0.5}
\definecolor{Keywords}{rgb}{0,0,1}
\definecolor{self}{rgb}{0,0,0}
\definecolor{Strings}{rgb}{0,0.63,0}
\definecolor{Comments}{rgb}{0,0.63,1}
\definecolor{Backquotes}{rgb}{0,0,0}
\definecolor{Classname}{rgb}{0,0,0}
\definecolor{FunctionName}{rgb}{0,0,0}
\definecolor{Operators}{rgb}{0,0,0}
\definecolor{Background}{rgb}{0.98,0.98,0.98}
\lstdefinelanguage{Python}{
	numbers=left,
	numberstyle=\footnotesize,
	numbersep=1em,
	xleftmargin=1em,
	framextopmargin=2em,
	framexbottommargin=2em,
	showspaces=false,
	showtabs=false,
	showstringspaces=false,
	frame=l,
	tabsize=4,
	% Basic
	basicstyle=\ttfamily\small\setstretch{1},
	backgroundcolor=\color{Background},
	% Comments
	commentstyle=\color{Comments}\slshape,
	% Strings
	stringstyle=\color{Strings},
	morecomment=[s][\color{Strings}]{"""}{"""},
	morecomment=[s][\color{Strings}]{'''}{'''},
	% keywords
	morekeywords={import,from,class,def,for,while,if,is,in,elif,else,not,and,or,print,break,continue,return,True,False,None,access,as,,del,except,exec,finally,global,import,lambda,pass,print,raise,try,assert},
	keywordstyle={\color{Keywords}\bfseries},
	% additional keywords
	morekeywords={[2]@invariant,pylab,numpy,np,scipy},
	keywordstyle={[2]\color{Decorators}\slshape},
	emph={self},
	emphstyle={\color{self}\slshape},
	%
}

\usepackage{import}

\newcommand{\uline}[1]{\rule[0pt]{#1}{0.4pt}}

\usepackage[para]{footmisc}

%------------------------------------------------
% Colors

%------------------------------------------------

%------------------------------------------------
%------------------------------------------------

%------------------------------------------------
% Fonts
\usepackage[T1]{fontenc}	 % For correct hyphenation and T1 encoding
\usepackage{lmodern} % Default font: latin modern font
%\usepackage{fourier} % Alternative font: utopia
%\usepackage{charter} % Alternative font: low-resolution roman font
\renewcommand{\familydefault}{\sfdefault} % Sans serif - this may need to be commented to see the alternative fonts
%------------------------------------------------

%------------------------------------------------
% Various required packages
\usepackage{amsthm} % Required for theorem environments
\usepackage{bm} % Required for bold math symbols (used in the footer of the slides)
\usepackage{graphicx} % Required for including images in figures
\usepackage{tikz} % Required for colored boxes
\usepackage{booktabs} % Required for horizontal rules in tables
\usepackage{multicol} % Required for creating multiple columns in slides
\usepackage{lastpage} % For printing the total number of pages at the bottom of each slide
\usepackage[english]{babel} % Document language - required for customizing section titles
\usepackage{microtype} % Better typography
\usepackage{tocstyle} % Required for customizing the table of contents


%\newtheorem{defi}{Definition}[section]
%\newtheorem{exmp}{Exercise}[section] %Label for examples
\newtheorem{remark}[theorem]{Remark} % Label for remarks
\newtheorem{algorithm}[theorem]{Algorithm} % Label for algorithms
\makeatletter % Correct qed adjustment
%------------------------------------------------

%------------------------------------------------
% The code for the box which can be used to highlight an element of a slide (such as a theorem)
\newcommand*{\mybox}[2]{ % The box takes two arguments: width and content
\par\noindent
\begin{tikzpicture}[mynodestyle/.style={rectangle,draw=Black,thick,inner sep=1.5mm, text justified,top color=white,bottom color=white,above}]\node[mynodestyle,at={(0.5*#1+2mm+0.4pt,0)}]{ % Box formatting
\begin{minipage}[t]{#1}
#2
\end{minipage}
};
\end{tikzpicture}
\par\vspace{-1.3em}}
%------------------------------------------------

%------------------------------------------------
% MODIFICATIONS BY JUSTIN STEVENS
%------------------------------------------------

\usepackage[nodayofweek,level]{datetime}
\usepackage{caption}
\usepackage{subcaption}
\usepackage{hyperref}
\newcommand{\pmid}{\mid\!\mid}
\usepackage{seqsplit}
\usepackage{amsfonts}
\usepackage{float} %use H to force it in place
\usepackage{amssymb} %for nmid
%\usepackage{enumitem} %for itemized lists with stars
\usepackage{amsmath}
\DeclareMathOperator{\lcm}{lcm}
%\usepackage{epigraph}
\usepackage{csquotes}
\usepackage{relsize}
\newcommand{\x}{\color{red}X\color{black}}

\usetikzlibrary{tikzmark}

\usepackage{textcomp}
\newcommand{\ballgolftikz}[1]{%
	\foreach \i  in {0,...,\number\numexpr#1 - 1\relax}{% 
		\pgfmathsetmacro\k{\i*sqrt(3)/2}
		\begin{scope}[shift={(\i*.5 cm,\k cm)}]
			\foreach \t in {1,...,\number\numexpr #1-\i\relax}{
				\shade[ball color= gray] (\t,0) circle (.5cm);}
	\end{scope}}
}  

\usepackage{pifont}
\usepackage{marginnote}
\reversemarginpar
\newcommand{\prechili}{\vspace*{1.2em}\hspace*{1.0em}}
\newcommand{\nochili}{\hspace*{3.8em}}
\newcommand{\chili}{\includegraphics[width=1.0em]{images/chili.png}}
\newcommand{\gim}{\marginnote{\chili}}
\newcommand{\yod}{\marginnote{\chili\chili}}
\newcommand{\kurumi}{\marginnote{\chili\chili\chili}}
\newcommand{\pencil}{\prechili\marginnote{\bfseries\ding{48}}}
\newcommand{\defi}{{\bfseries\color{ForestGreen}Definition. }}

\usepackage{mathtools}

\newenvironment{polyalign}[1][9]
{\array{c*{#1}{@{}>{{}}c<{{}}@{}c@{}}}}
{\endarray}

\usepackage{tabularx}
\usepackage{bm}
\usepackage{mwe}% provides example images (when installed)
\newcommand\measureISpecification{6ex}% not defined in mwe
\newcommand{\ctab}[1]{\raisebox{\dimexpr \measureISpecification/2 -.748ex}{#1}}% vertically centers numbers

\usetikzlibrary{tikzmark}

\usetikzlibrary{arrows} 

\usepgflibrary{fpu}
\usetikzlibrary{positioning}
\usepackage{thmtools}
\theoremstyle{definition}
\declaretheorem[name=\bfseries Problem]{prob}
%\declaretheorem[name=\bfseries Example]{exmp}
\theoremstyle{plain}

\newenvironment{soln}{\begin{myproof}[Solution]}{\end{myproof}}
\newcommand*\circled[1]{\tikz[baseline=(char.base)]{% <---- BEWARE
		\node[shape=circle,draw,inner sep=2pt] (char) {#1};}}
	
\usepackage{comment}
\usepackage{systeme}


\usepackage{scalerel}
\usepackage{stackengine}
\newcommand\showdiv[1]{\overline{\smash{\hstretch{.5}{)}\mkern-3.2mu\hstretch{.5}{)}}#1}}
\newcommand\ph[1]{\textcolor{white}{#1}}


\usepackage{animate}
\usepackage{enumerate}
\setbeamertemplate{title page}
{
  \vbox{}
  \begingroup
    \centering
    {\usebeamercolor[fg]{titlegraphic}\inserttitlegraphic\par}\vskip1em
    \begin{beamercolorbox}[sep=8pt,center]{title}
      \usebeamerfont{title}\inserttitle\par%
      \ifx\insertsubtitle\@empty%
      \else%
        \vskip0.25em%
        {\usebeamerfont{subtitle}\usebeamercolor[fg]{subtitle}\insertsubtitle\par}%
      \fi%
    \end{beamercolorbox}%
    \vskip1em\par
    \begin{beamercolorbox}[sep=8pt,center]{author}
      \insertauthor
    \end{beamercolorbox}
        \vskip1em\par

  \endgroup
  \vfill
}
